\documentclass[a4paper]{book}
\usepackage{fullpage}
\usepackage{verbatim}
\usepackage{listings}

\usepackage[ruled,vlined]{algorithm2e}

\newcommand{\forcond}{$i=0$ \KwTo $n$}

\usepackage{hyperref}
\setcounter{tocdepth}{1}


\begin{document}

\title{{\Huge ns3 for beginners} \\
\vspace{1em}\large A guide towards running Hyrax simulations\vspace{5em}}
\author{Joaquim M. E. Silva \\ jqmmes@gmail.com}
\date{December 2015}
\maketitle

\tableofcontents

\chapter{Introduction}

\section{Installing}

\section{Configurations}

\section{Running a simulation}


\chapter{Simulations}

\section{Overlay Simulation}

\subsection{Objectives}


\subsection{Parameters}

\subsection{Examples}



\section{Technologies Experiment}


\subsection{Objectives}


\subsection{Parameters}

\begin{itemize}
	\item[] Running:
	\begin{itemize}
		\item[] \footnotesize \begin{verbatim}./waf --run="scratch/Experiment/Experiment --Nodes=1 --Servers=1 --Scenario=3 --Seed=$RANDOM --ExclusiveServers"\end{verbatim}
	\end{itemize}
\end{itemize}

\begin{itemize}
	\item[] Parameters:
	\begin{itemize}
		\item[] \textbf{Nodes}: Number of Nodes to be used in the simulation
		\item[] \textbf{Servers}: Number of Servers to be used in the simulation
		\item[] \textbf{Scenario}: Scenario to run
		\begin{itemize}
			\item 1: 1 Server + AP + n Nodes
			\item 2: AP + m Mobile Servers + n Nodes (m<=n)
			\item 3: AP + TDLS + m Mobile Servers + n Nodes (m<=n)
			\item 41: WD + GO as Server + n Nodes
			\item 42: WD + GO + m Mobile Servers + n Nodes (m<=n)
			\item 43: WD + m Mobiles Servers + n Nodes (m<=n) No groups formed in the beggining
			\item 51: WD + Legacy AP as Server + n Nodes
			\item 52: WD + Legacy AP + m Mobile Servers + n Nodes (m<=n)
			\item 6: WD + GO + TDLS + m Mobile Servers + n Nodes (m<=n)
		\end{itemize}
		\item[] \textbf{FileSize}: File Size to be shared
		\item[] \textbf{Debug}: Debug socket callbacks
		\item[] \textbf{ShowPackets}: Show every packet received
		\item[] \textbf{ShowData}: Show Send/Receive instead of the time a transfer took
		\item[] \textbf{Seed}: Seed to be used
		\item[] \textbf{ExclusiveServers}: Use Exclusive Server. (Server Don't act as Client)
		\item[] \textbf{SegmentSize}: TCP Socket Segment Size
	\end{itemize}
\end{itemize}	

\subsection{Examples}



\section{CMU Review App Simulation}

\subsection{Objectives}


\subsection{Parameters}

\subsection{Examples}

\chapter{Post-Processing}

\section{Overlay Simulation}

\section{Technologies Experiment}

\section{CMU Review App Simulation}

\chapter{Code}

\section{Network Configurations}


\section{\textit{VirtualDiscovery}}

\begin{itemize}
\item[] \textbf{Public Methods:}
\item[]\begin{verbatim}
void VirtualDiscovery::add(Ipv4Address ip, uint16_t port)
tuple<Ipv4Address,uint16_t> VirtualDiscovery::discover(void)
vector<tuple<Ipv4Address,uint16_t>> VirtualDiscovery::getAll(void)
uint32_t VirtualDiscovery::GetN(void)
void VirtualDiscovery::remove(Ipv4Address ip, uint16_t port)
\end{verbatim}
\end{itemize}

\section{TDLS}
\begin{itemize}
\item[] \textbf{Public Methods:}
\item[]\begin{verbatim}
void SendTDLS(Ipv4Address ip, uint16_t port, std::string message)
\end{verbatim}
\item[] \textbf{Algorithms:}
\end{itemize}

\begin{algorithm}[H]
\SetKwFunction{STDLS}{SendTDLS}
\SetKwProg{Fn}{Function}{}{end}
\SetKwFunction{Schedule}{Schedule}
\SetKwFunction{CheckTDLS}{CheckTDLS}
\SetKwProg{callback}{Callback}{}{end}
\SetKwFunction{ConnectSuccess}{ConnectSuccess}
\SetKwFunction{DeleteEntry}{DeleteEntry}
\SetKwFunction{assert}{assert}
\SetKwArray{TDLSData}{TDLSData}
\SetAlgoLined


\KwData{$message$ - Message to be sent; $socket$ - TDLS (using Wi-Fi Ad-hoc) socket}
\KwResult{A message is sent using TDLS or regular Wifi as fallback}
\SetKwInOut{Input}{Input}\SetKwInOut{Output}{Output}

\Input{$timeout$ - duration until CheckTDLS fallback occurs}
\Output{nothing}

\Fn(\tcc*[h]{Algorithm to Send a message with TDLS}){\STDLS{socket, message}}{
	\uIf{ActiveTDLSCons $<$ MAX}{
		\nl $socket \rightarrow connect(ServerIp)$\;
		\nl $\TDLSData{socket} \gets socket$ \;
		\nl $\TDLSData{message} \gets message$ \;
		\nl $\TDLSData{delivered} \gets$ false \;
		\nl $ActiveTDLSCons++$ \;
		\nl \Schedule{\CheckTDLS{socket, message}, timeout}\;
	}
	\Else{
		\nl $RegularSocket \rightarrow connect(ServerIp)$\;
		\nl $RegularSocket \rightarrow send(message)$\;
	}
}

\Fn(){\CheckTDLS{socket, message}}{
	\If(){$\TDLSData{socket} = socket \wedge \TDLSData{delivered} = false$}{
		\nl $RegularSocket \rightarrow connect(ServerIp)$\;
		\nl $RegularSocket \rightarrow send(message)$\;
	}
	\tcc*[h]{Deletes Hashmap entry} \\
	\nl \DeleteEntry{\TDLSData{$socket$}} \;
	
}

\callback(\tcc*[h]{Callback called if $socket \rightarrow connect(ServerIp)$ succeeds}){\ConnectSuccess(socket)}{
	\nl \TDLSData{$delivered$} $\gets$ true \;
}

\caption{TDLS ns3 Algorithm - Client\label{tdlsAlgC}}
\end{algorithm}

\begin{algorithm}[H]
\SetKwFunction{TDLSA}{TDLSAccept}
\SetKwFunction{SPA}{SecondPhaseAccept}
\SetKwProg{Fn}{Function}{}{end}
\SetKwProg{callback}{Callback}{}{end}
\SetAlgoLined

\KwData{$socket$ - TDLS (using Wi-Fi Ad-hoc) socket}
\KwResult{A message is received using TDLS or regular Wifi as fallback}
\SetKwInOut{Input}{Input}\SetKwInOut{Output}{Output}

\Input{$MAX$ - Maximum number of simultaneous TDLS sockets opened}
\Output{nothing}


\Fn(){\TDLSA{ListenSocket}}{
	\If(){$ActiveTDLSCons < MAX$}{
		%\nl $ConnectedTDLS \gets socket$ \;
		\nl $ActiveTDLSCons++$ \;
		\nl $socket \rightarrow setAcceptCallback(\SPA)$ \;
		\nl \Return true\;
	}
	\nl \Return false\;
}

\callback(){\SPA{socket}}{
	\nl $ConnectedTDLS \gets socket$\;
}




\caption{TDLS ns3 Algorithm - Server\label{tdlsAlgS}}
\end{algorithm}

\begin{algorithm}[H]
\SetKwProg{Fn}{Function}{}{end}
\SetKwFunction{Close}{CloseSocket}
\SetAlgoLined

\Fn(){\Close{socket}}{
	\nl $socket \rightarrow Close()$ \;
	\If(){$ConnectedTDLS = socket$}{
		\nl $ActiveTDLSCons--$ \;
		\nl $ConnectedTDLS \gets Null$ \;
	}
}

\caption{TDLS ns3 Algorithm - Closing Socket\label{tdlsAlgS}}
\end{algorithm}

\section{Wifi-Direct}

\section{Mobility}

\chapter{Advanced}

\section{Developing a new simulation from scratch}


\section{Tracing}

\section{Parallel Execution}


\section{Direct Code Execution}

\bibliographystyle{plain}
\nocite{2015ns3_manual}
\nocite{lacage2006yet}
\bibliography{guide} 

\end{document}